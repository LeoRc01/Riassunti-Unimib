% the first part of the document before the begin is called preamble
\documentclass[12pt, a4paper]{article}
\usepackage{graphicx}
\usepackage{hyperref}
\graphicspath{ {./images/} }
%
\author{Leonardo Valente}
\title{Mini4wd App}

\begin{document}
    \maketitle

    L'idea alla base dell'applicazione è quella di creare una specie di "Social Network"
    per i setup delle Mini4WD. L'utene ha la possibilità di creare il proprio setup, ma il sistema non dovrà
    porre alcuna limitazione. Infatti, sarà l'utente finale a creare i propri campi di interesse per offrire una
    completa esperienza di customizzazione.
    Ci dovrà essere la possibilità di caricare molteplici foto per ogni setup, in modo tale da far capire a chi legge 
    come dovrà venire il modello.
    \\Il setup può avere più versioni di se stesso, mantenendo uno storico delle modifiche. 
    \\ \textbf{Ad esempio:} 
    \begin{itemize}
        \item Preparo un setup a casa e vado in pista
        \item Faccio correre la macchina e vedo che non va bene
        \item Faccio un cambiamento (es. sposto i pesi più in avanti) e aggiorno il setup
        \item \textit{Verrà creata in automatico una V2 del setup}
    \end{itemize}
    E così via...
    \\
    Si potrà rivisitare tutto lo storico delle modifiche ed eventualmente eliminare una versione specifica.
    \\\\
    L'utente disporrà di un profilo dove (volendo) potrà condividere i propri setup (una sola versione per ogni setup, la migliore)
    o eventualmente creare un PDF con i dettagli del modello da condividere con chi non dispone dell'applicazione.
    \\\\
    Ci sarà la possiblità di cercare gli altri utenti tramite \textbf{nome utente} e creare un sistema di 
    Follow/Unfollow proprio come se fosse un social network. I setup delle persone che seguiremo ci verranno mostrati 
    all'interno della nostra Homepage.

\end{document}